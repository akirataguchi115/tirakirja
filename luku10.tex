\chapter{Verkkojen perusteet}

\index{verkko}

Voimme ratkaista monia algoritmiikan ongelmia
esittämällä tilanteen \emph{verkkona} (\emph{graph}) ja käyttämällä sitten
sopivaa verkkoalgoritmia.
Tyypillinen esimerkki verkosta on tieverkosto,
joka muodostuu kaupungeista ja niiden välisistä teistä.
Tällaisessa verkossa ongelmana voi olla selvittää vaikkapa,
kuinka voimme matkustaa kaupungista $a$ kaupunkiin $b$.

Tässä luvussa aloitamme verkkoihin tutustumisen
käymällä läpi verkkojen käsitteitä sekä tapoja
esittää verkkoja ohjelmoinnissa.
Tämän jälkeen näemme, miten voimme tutkia verkkojen rakennetta
ja ominaisuuksia syvyyshaun ja leveyshaun avulla.
Seuraavissa kirjan luvuissa jatkamme verkkojen käsittelyä ja
opimme lisää verkkoalgoritmeja.

\section{Verkkojen käsitteitä}

\index{solmu}
\index{kaari}

Verkko muodostuu \emph{solmuista} (\emph{node} tai \emph{vertex}) ja
niiden välisistä \emph{kaarista} (\emph{edge}).
Kuvassa \ref{fig:veresi} on verkko,
jossa on viisi solmua ja seitsemän kaarta.
Merkitsemme verkon solmujen
määrää kirjaimella $n$ ja kaarten määrää
kirjaimella $m$.
Lisäksi numeroimme verkon solmut kokonaisluvuin
$1,2,\dots,n$.

\begin{figure}
\center
\begin{center}
\begin{tikzpicture}[scale=0.6]
\small
\node[draw, circle] (1) at (1,3) {$1$};
\node[draw, circle] (2) at (4,3) {$2$};
\node[draw, circle] (3) at (1,1) {$3$};
\node[draw, circle] (4) at (4,1) {$4$};
\node[draw, circle] (5) at (6,2) {$5$};

\path[draw,thick,-] (1) -- (2);
\path[draw,thick,-] (1) -- (3);
\path[draw,thick,-] (1) -- (4);
\path[draw,thick,-] (3) -- (4);
\path[draw,thick,-] (2) -- (4);
\path[draw,thick,-] (2) -- (5);
\path[draw,thick,-] (4) -- (5);
\end{tikzpicture}
\end{center}
\caption{Verkko, jossa on viisi solmua ja seitsemän kaarta.}
\label{fig:veresi}
\end{figure}

\index{naapuri}
\index{aste}

Kaksi solmua ovat \emph{vierekkäin} (\emph{adjacent}),
jos niiden välillä on kaari.
Solmun \emph{naapureja} (\emph{neighbor}) ovat kaikki solmut,
joihin se on yhteydessä kaarella,
ja solmun \emph{aste} (\emph{degree}) on sen naapureiden määrä.
Kuvassa \ref{fig:veresi} solmun 2 naapurit ovat 1, 4 ja 5,
joten solmun aste on 3.

\subsubsection{Polku ja sykli}

\begin{figure}
\center
\begin{center}
\begin{tikzpicture}[scale=0.6]
\small
\begin{scope}
\node[draw, circle] (1) at (1,3) {$1$};
\node[draw, circle] (2) at (4,3) {$2$};
\node[draw, circle] (3) at (1,1) {$3$};
\node[draw, circle] (4) at (4,1) {$4$};
\node[draw, circle] (5) at (6,2) {$5$};

\path[draw,thick,-] (1) -- (2);
\path[draw,thick,-] (1) -- (3);
\path[draw,thick,-] (1) -- (4);
\path[draw,thick,-] (3) -- (4);
\path[draw,thick,-] (2) -- (4);
\path[draw,thick,-] (2) -- (5);
\path[draw,thick,-] (4) -- (5);

\path[draw,thick,->,red,line width=2pt] (1) -- (2);
\path[draw,thick,->,red,line width=2pt] (2) -- (5);
\end{scope}
\begin{scope}[xshift=8cm]
\node[draw, circle] (1) at (1,3) {$1$};
\node[draw, circle] (2) at (4,3) {$2$};
\node[draw, circle] (3) at (1,1) {$3$};
\node[draw, circle] (4) at (4,1) {$4$};
\node[draw, circle] (5) at (6,2) {$5$};

\path[draw,thick,-] (1) -- (2);
\path[draw,thick,-] (1) -- (3);
\path[draw,thick,-] (1) -- (4);
\path[draw,thick,-] (3) -- (4);
\path[draw,thick,-] (2) -- (4);
\path[draw,thick,-] (2) -- (5);
\path[draw,thick,-] (4) -- (5);

\path[draw,thick,->,red,line width=2pt] (1) -- (3);
\path[draw,thick,->,red,line width=2pt] (3) -- (4);
\path[draw,thick,->,red,line width=2pt] (4) -- (5);
\end{scope}
\end{tikzpicture}
\end{center}
\caption{Kaksi polkua solmusta $1$ solmuun $5$.}
\label{fig:verpoe}
\end{figure}

\index{polku}

Verkossa oleva \emph{polku} (\emph{path}) on kaaria pitkin kulkeva reitti
lähtösolmusta kohdesolmuun.
Kuva \ref{fig:verpoe} näyttää kaksi
mahdollista polkua solmusta $1$ solmuun $5$.
Polut ovat $1 \rightarrow 2 \rightarrow 5$
ja $1 \rightarrow 3 \rightarrow 4 \rightarrow 5$.

\begin{figure}
\center
\begin{center}
\begin{tikzpicture}[scale=0.6]
\small
\node[draw, circle] (1) at (1,3) {$1$};
\node[draw, circle] (2) at (4,3) {$2$};
\node[draw, circle] (3) at (1,1) {$3$};
\node[draw, circle] (4) at (4,1) {$4$};
\node[draw, circle] (5) at (6,2) {$5$};

\path[draw,thick,-] (1) -- (2);
\path[draw,thick,-] (1) -- (3);
\path[draw,thick,-] (1) -- (4);
\path[draw,thick,-] (3) -- (4);
\path[draw,thick,-] (2) -- (4);
\path[draw,thick,-] (2) -- (5);
\path[draw,thick,-] (4) -- (5);

\path[draw,thick,->,red,line width=2pt] (2) -- (4);
\path[draw,thick,->,red,line width=2pt] (4) -- (5);
\path[draw,thick,->,red,line width=2pt] (5) -- (2);
\end{tikzpicture}
\end{center}
\caption{Verkossa on sykli $2 \rightarrow 4 \rightarrow 5 \rightarrow 2$.}
\label{fig:versyk}
\end{figure}

\index{sykli}
\index{syklitön verkko}

Polku on \emph{sykli} (\emph{cycle}), jos sen alku- ja loppusolmu on sama,
siinä on ainakin yksi kaari 
eikä se kulje kahta kertaa saman solmun tai kaaren kautta.
Kuvassa \ref{fig:versyk} on esimerkkinä sykli
$2 \rightarrow 4 \rightarrow 5 \rightarrow 2$.
Jos verkossa ei ole yhtään sykliä, verkko on \emph{syklitön}
(\emph{acyclic}).

\subsubsection{Yhtenäisyys ja komponentit}

\index{yhtenäinen verkko}

Verkko on \emph{yhtenäinen} (\emph{connected}),
jos minkä tahansa kahden solmun välillä on polku.
Kuvan \ref{fig:veresi} verkko on yhtenäinen,
mutta kuvan \ref{fig:veryht} verkko ei ole yhtenäinen,
koska esimerkiksi solmujen $1$ ja $2$ välillä ei ole polkua.

\index{komponentti}
\index{yhtenäinen komponentti}

Verkko voidaan esittää aina kokoelmana yhtenäisiä \emph{komponentteja}
(\emph{component}).
Kuvassa \ref{fig:veryht} yhtenäiset komponentit
ovat $\{1,3\}$ ja $\{2,4,5\}$.

\begin{figure}
\center
\begin{center}
\begin{tikzpicture}[scale=0.6]
\small
\node[draw, circle] (1) at (1,3) {$1$};
\node[draw, circle] (2) at (4,3) {$2$};
\node[draw, circle] (3) at (1,1) {$3$};
\node[draw, circle] (4) at (4,1) {$4$};
\node[draw, circle] (5) at (6,2) {$5$};

\path[draw,thick,-] (1) -- (3);
\path[draw,thick,-] (2) -- (4);
\path[draw,thick,-] (2) -- (5);
\path[draw,thick,-] (4) -- (5);
\end{tikzpicture}
\end{center}
\caption{Verkon yhtenäiset komponentit ovat $\{1,3\}$ ja $\{2,4,5\}$.}
\label{fig:veryht}
\end{figure}

\index{puu}

Verkko on \emph{puu} (\emph{tree}), jos se on sekä yhtenäinen
että syklitön.
Puussa kaarten määrä on aina yhden pienempi
kuin solmujen määrä, ja jokaisen kahden solmun
välillä on yksikäsitteinen polku.
Kuvassa \ref{fig:verpuu} on esimerkkinä puu,
jossa on viisi solmua ja neljä kaarta.

\begin{figure}
\center
\begin{center}
\begin{tikzpicture}[scale=0.6]
\small
\node[draw, circle] (1) at (1,3) {$1$};
\node[draw, circle] (2) at (4,3) {$2$};
\node[draw, circle] (3) at (1,1) {$3$};
\node[draw, circle] (4) at (4,1) {$4$};
\node[draw, circle] (5) at (6,2) {$5$};

\path[draw,thick,-] (1) -- (3);
\path[draw,thick,-] (3) -- (4);
\path[draw,thick,-] (2) -- (4);
\path[draw,thick,-] (4) -- (5);
\end{tikzpicture}
\end{center}
\caption{Puu eli yhtenäinen, syklitön verkko.}
\label{fig:verpuu}
\end{figure}

\subsubsection{Suunnattu verkko}

\index{suunnattu verkko}

Kun verkko on \emph{suunnattu} (\emph{directed}),
jokaisella kaarella on tietty suunta,
jonka mukaisesti kaarta pitkin tulee kulkea.
Suunnat rajoittavat siis verkossa liikkumista.
Kuvassa \ref{fig:versuu} on esimerkkinä
suunnattu verkko, jossa
voimme kulkea solmusta $1$ solmuun $5$
polkua $1 \rightarrow 2 \rightarrow 5$,
mutta emme voi kulkea mitenkään solmusta $5$ solmuun $1$.

\subsubsection{Painotettu verkko}

\index{painotettu verkko}

Kun verkko on \emph{painotettu} (\emph{weighted}),
jokaiseen kaareen liittyy jokin paino,
joka kuvaa tyypillisesti kaaren pituutta.
Kun kuljemme polkua painotetussa verkossa,
polun pituus on kaarten painojen summa.
Kuvassa \ref{fig:verpae} on esimerkkinä painotettu verkko,
jossa polun $1 \rightarrow 2 \rightarrow 5$
pituus on $5+8=13$ ja polun
$1 \rightarrow 3 \rightarrow 4 \rightarrow 5$
pituus on $2+4+3=9$.

\begin{figure}[h]
\center
\begin{center}
\begin{tikzpicture}[scale=0.6]
\small
\node[draw, circle] (1) at (1,3) {$1$};
\node[draw, circle] (2) at (4,3) {$2$};
\node[draw, circle] (3) at (1,1) {$3$};
\node[draw, circle] (4) at (4,1) {$4$};
\node[draw, circle] (5) at (6,2) {$5$};

\path[draw,thick,->] (1) -- (2);
\path[draw,thick,->] (1) -- (3);
\path[draw,thick,->] (1) -- (4);
\path[draw,thick,->] (3) -- (4);
\path[draw,thick,->] (2) -- (4);
\path[draw,thick,->] (2) -- (5);
\path[draw,thick,->] (4) -- (5);
\end{tikzpicture}
\end{center}
\caption{Suunnattu verkko.}
\label{fig:versuu}
\end{figure}

\begin{figure}[h]
\center
\begin{center}
\begin{tikzpicture}[scale=0.6,label distance=-1.5mm]
\small
\node[draw, circle] (1) at (1,3) {$1$};
\node[draw, circle] (2) at (4,3) {$2$};
\node[draw, circle] (3) at (1,1) {$3$};
\node[draw, circle] (4) at (4,1) {$4$};
\node[draw, circle] (5) at (6,2) {$5$};

\path[draw,thick,-] (1) -- node[font=\small,label=above:5] {} (2);
\path[draw,thick,-] (1) -- node[font=\small,label=left:2] {} (3);
\path[draw,thick,-] (1) -- node[font=\small,label=above:9] {} (4);
\path[draw,thick,-] (3) -- node[font=\small,label=below:4] {} (4);
\path[draw,thick,-] (2) -- node[font=\small,label=above:8] {} (5);
\path[draw,thick,-] (4) -- node[font=\small,label=right:2] {} (2);
\path[draw,thick,-] (4) -- node[font=\small,label=below:3] {} (5);
\end{tikzpicture}
\end{center}
\caption{Painotettu verkko.}
\label{fig:verpae}
\end{figure}

\subsubsection{Yksinkertainen verkko}

\index{yksinkertainen verkko}

Usein oletuksena on, että verkko on \emph{yksinkertainen} (\emph{simple}),
jolloin siinä ei ole kahta samanlaista kaarta ja jokainen kaari yhdistää
kaksi eri solmua. Kaikki tähän mennessä kirjassa esitetyt verkot ovat
olleet yksinkertaisia.

Kuvassa \ref{fig:vereiy} on esimerkkinä verkko, joka ei ole yksinkertainen.
Tähän on kaksi syytä: solmujen $1$ ja $2$ välillä on kaksi kaarta
ja lisäksi solmusta $5$ on kaari itseensä.

\begin{figure}
\center
\begin{center}
\begin{tikzpicture}[scale=0.6]
\small
\node[draw, circle] (1) at (1,3) {$1$};
\node[draw, circle] (2) at (4,3) {$2$};
\node[draw, circle] (3) at (1,1) {$3$};
\node[draw, circle] (4) at (4,1) {$4$};
\node[draw, circle] (5) at (6,2) {$5$};

\path[draw,thick,-] (1) edge [bend right=30] (2);
\path[draw,thick,-] (2) edge [bend right=30] (1);
\path[draw,thick,-] (1) -- (3);
\path[draw,thick,-] (3) -- (4);
\path[draw,thick,-] (2) -- (4);
\path[draw,thick,-] (4) -- (5);
%\path[every node/.style={}] (5) edge[loop] (5);

\path[-,every loop/.style={looseness=6}] (5)
         edge [in=120,out=60,loop] ();

\end{tikzpicture}
\end{center}
\caption{Verkko, joka ei ole yksinkertainen.}
\label{fig:vereiy}
\end{figure}

\section{Verkot ohjelmoinnissa}

Verkon esittämiseen ohjelmoinnissa on monia mahdollisuuksia.
Sopivan esitystavan valintaan vaikuttaa,
miten haluamme käsitellä verkkoa algoritmissa,
koska jokaisessa esitystavassa on omat etunsa.
Seuraavaksi käymme läpi kolme tavallista esitystapaa.

\begin{figure}[h]
\center
\begin{center}
\begin{tikzpicture}[scale=0.6]
\small
\begin{scope}
\node[draw, circle] (1) at (1,3) {$1$};
\node[draw, circle] (2) at (4,3) {$2$};
\node[draw, circle] (3) at (1,1) {$3$};
\node[draw, circle] (4) at (4,1) {$4$};
\node[draw, circle] (5) at (6,2) {$5$};

\path[draw,thick,->] (1) -- (2);
\path[draw,thick,->] (1) -- (3);
\path[draw,thick,->] (1) -- (4);
\path[draw,thick,->] (3) -- (4);
\path[draw,thick,->] (2) -- (4);
\path[draw,thick,->] (2) -- (5);
\path[draw,thick,->] (4) -- (5);
\end{scope}
\begin{scope}[xshift=9cm,yshift=-0.5cm]
\draw (0,0) grid (1,5);
\foreach \x in {1,...,5} \node at (0.5,0.5-\x+5) {\x};
\draw[->,thick] (1,4.5) -- (2,4.5);
\draw[->,thick] (1,3.5) -- (2,3.5);
\draw[->,thick] (1,2.5) -- (2,2.5);
\draw[->,thick] (1,1.5) -- (2,1.5);
\draw (2,4.1) rectangle (2.8,4.9);
\draw (2.8,4.1) rectangle (3.6,4.9);
\draw (3.6,4.1) rectangle (4.4,4.9);
\draw (2,3.1) rectangle (2.8,3.9);
\draw (2.8,3.1) rectangle (3.6,3.9);
\draw (2,2.1) rectangle (2.8,2.9);
\draw (2,1.1) rectangle (2.8,1.9);
\node at (2.4,4.5) {$2$};
\node at (3.2,4.5) {$3$};
\node at (4.0,4.5) {$4$};
\node at (2.4,3.5) {$4$};
\node at (3.2,3.5) {$5$};
\node at (2.4,2.5) {$4$};
\node at (2.4,1.5) {$5$};
\end{scope}
\end{tikzpicture}
\end{center}
\caption{Verkon vieruslistaesitys.}
\label{fig:vervil}
\end{figure}

\subsection{Vieruslistaesitys}

\index{vieruslista}

Tavallisin tapa esittää verkko on luoda kullekin solmulle
\emph{vieruslista} (\emph{adjacency list}), joka kertoo, mihin solmuihin voimme
siirtyä solmusta kaaria pitkin.
Kuvassa \ref{fig:vervil} on esimerkkinä verkko
ja sitä vastaava vieruslistaesitys.
Voimme luoda vieruslistat näin ohjelmoinnissa:

\begin{code}
verkko[1].add(2)
verkko[1].add(3)
verkko[1].add(4)
verkko[2].add(4)
verkko[2].add(5)
verkko[3].add(4)
verkko[4].add(5)
\end{code}

Vieruslistaesitys on monessa tilanteessa hyvä tapa tallentaa verkko,
koska haluamme usein selvittää,
mihin solmuihin pääsemme siirtymään tietystä solmusta kaaria pitkin.
Esimerkiksi seuraava koodi käy läpi solmut,
joihin voimme siirtyä solmusta $a$ kaarella:

\begin{code}
for b in verkko[a]
    // käsittele solmu b
\end{code}

Suuntaamaton verkko voidaan esittää vieruslistoina niin,
että jokainen kaari tallennetaan molempiin suuntiin.
Jos taas verkko on painotettu, voimme tallentaa jokaisesta
kaaresta sekä kohdesolmun että painon.

\subsection{Kaarilistaesitys}

\index{kaarilista}

Toinen tapa tallentaa verkko on luoda \emph{kaarilista}
(\emph{edge list}),
joka sisältää kaikki verkon kaaret.
Voisimme luoda esimerkkiverkon kaarilistan näin:

\begin{code}
kaaret.add((1,2))
kaaret.add((1,3))
kaaret.add((1,4))
kaaret.add((2,4))
kaaret.add((2,5))
kaaret.add((3,4))
kaaret.add((4,5))
\end{code}

Tässä tapauksessa jokainen listan alkio on pari,
jossa on kaaren alku- ja loppusolmu.
Vastaavasti voisimme tallentaa myös kaaret kolmikkoina,
joissa on lisäksi mukana kaaren paino.

Kaarilista on hyvä esitystapa algoritmeissa,
joissa tulee pystyä käy\-mään helposti läpi
kaikki verkon kaaret eikä ole tarvetta
selvittää tietystä solmusta lähteviä kaaria.

\subsection{Vierusmatriisiesitys}

\begin{figure}
\center
\begin{center}
\begin{tikzpicture}[scale=0.6]
\small
\begin{scope}
\node[draw, circle] (1) at (1,3) {$1$};
\node[draw, circle] (2) at (4,3) {$2$};
\node[draw, circle] (3) at (1,1) {$3$};
\node[draw, circle] (4) at (4,1) {$4$};
\node[draw, circle] (5) at (6,2) {$5$};

\path[draw,thick,->] (1) -- (2);
\path[draw,thick,->] (1) -- (3);
\path[draw,thick,->] (1) -- (4);
\path[draw,thick,->] (3) -- (4);
\path[draw,thick,->] (2) -- (4);
\path[draw,thick,->] (2) -- (5);
\path[draw,thick,->] (4) -- (5);
\end{scope}
\begin{scope}[xshift=9cm,yshift=-0.5cm]
\draw (0,0) grid (5,5);
\foreach \x in {1,...,5} \node at (-0.5,0.5-\x+5) {\x};
\foreach \x in {1,...,5} \node at (-0.5+\x,5.5) {\x};
\foreach \x/\v in {1/0,2/1,3/1,4/1,5/0} \node at (-0.5+\x,4.5) {\v};
\foreach \x/\v in {1/0,2/0,3/0,4/1,5/1} \node at (-0.5+\x,3.5) {\v};
\foreach \x/\v in {1/0,2/0,3/0,4/1,5/0} \node at (-0.5+\x,2.5) {\v};
\foreach \x/\v in {1/0,2/0,3/0,4/0,5/1} \node at (-0.5+\x,1.5) {\v};
\foreach \x/\v in {1/0,2/0,3/0,4/0,5/0} \node at (-0.5+\x,0.5) {\v};
\end{scope}
\end{tikzpicture}
\end{center}
\caption{Verkon vierusmatriisiesitys.}
\label{fig:vervim}
\end{figure}

\index{vierusmatriisi}

\emph{Vierusmatriisi} (\emph{adjacency matrix}) on kaksiulotteinen rakenne,
joka kertoo jokaisesta verkon kaaresta,
esiintyykö se verkossa.
Matriisin rivin $a$ sarakkeessa $b$ oleva arvo
ilmaisee, onko verkossa kaarta solmusta $a$ solmuun $b$.
Kuvassa \ref{fig:vervim} on esimerkki
verkon vierusmatriisiesityksestä.
Tässä tapauksessa matriisin jokainen arvo on
0 (ei kaarta) tai 1 (kaari).

Ohjelmoinnissa voimme tallentaa vierusmatriisin kaksiulotteisena
taulukkona tai listana. Tässä on esimerkkiverkkoa vastaava
vierusmatriisi:

\begin{code}
verkko[1][2] = 1
verkko[1][3] = 1
verkko[1][4] = 1
verkko[2][4] = 1
verkko[2][5] = 1
verkko[3][4] = 1
verkko[4][5] = 1
\end{code}

Jos verkko on painotettu, kaarten painot voidaan vastaavasti
merkitä vierusmatriisiin.
Vierusmatriisin etuna on, että siitä voi tarkastaa helposti,
onko tietty kaari verkossa.
Esitystapa kuluttaa kuitenkin paljon muistia,
eikä sitä voi käyttää, jos verkon solmujen määrä on suuri.

\section{Verkon läpikäynti}

Tutustumme seuraavaksi kahteen keskeiseen verkkoalgoritmiin,
jotka käyvät läpi verkossa olevia solmuja ja kaaria.
Ensin käsittelemme syvyyshakua, joka on yleiskäyttöinen algoritmi
verkon läpikäyntiin, ja sen jälkeen leveyshakua,
jonka avulla voimme löytää lyhimpiä polkuja verkossa.

\subsection{Syvyyshaku}

\index{syvyyshaku}

\emph{Syvyyshaku} (\emph{depth-first search} eli \emph{DFS})
on verkkojen käsittelyn yleistyökalu,
jonka avulla voimme selvittää monia asioita verkon rakenteesta.
Kun alamme tutkia verkkoa syvyyshaulla,
tulee päättää ensin, mistä solmusta haku lähtee liikkeelle.
Haku etenee vuorollaan kaikkiin solmuihin, jotka ovat saavutettavissa
lähtösolmusta kulkemalla kaaria pitkin.

Syvyyshaku pitää yllä jokaisesta verkon solmusta tietoa,
onko solmussa jo vierailtu.
Kun haku saapuu solmuun, jossa se ei ole vieraillut aiemmin,
se merkitsee solmun vierailluksi ja alkaa käydä
läpi solmusta lähteviä kaaria.
Jokaisen kaaren kohdalla haku etenee verkon niihin osiin,
joihin pääsee kaaren kautta.
Lopulta kun haku on käynyt läpi kaikki kaaret,
se perääntyy taaksepäin samaa reittiä kuin tuli solmuun.

Kuvassa \ref{fig:syvhak} on esimerkki syvyyshaun toiminnasta.
Jokaisessa vaiheessa harmaat solmut ovat solmuja,
joissa haku on jo vieraillut.
Tässä esimerkissä haku lähtee liikkeelle solmusta 1,
josta pääsee kaarella solmuihin 2 ja 3.
Haku etenee ensin solmuun 2, josta pääsee edelleen
solmuihin 4 ja 5.
Tämän jälkeen haku ei enää löydä uusia solmuja
tästä verkon osasta, joten se perääntyy takaisin
kulkemaansa reittiä solmuun 1.
Lopuksi haku käy vielä solmussa 3,
josta ei pääse muihin uusiin solmuihin.
Nyt haku on käynyt läpi kaikki solmut,
joihin pääsee solmusta 1.

\begin{figure}
\center
\begin{center}
\begin{tikzpicture}[scale=0.6]
\scriptsize
\newcommand\verkko[6]{
\node[draw, circle, fill=#2] (1) at (0,0) {$1$};
\node[draw, circle, fill=#3] (2) at (2,0) {$2$};
\node[draw, circle, fill=#4] (3) at (0,-2) {$3$};
\node[draw, circle, fill=#5] (4) at (2,-2) {$4$};
\node[draw, circle, fill=#6] (5) at (4,-1) {$5$};
\path[draw,thick,-] (1) -- (2);
\path[draw,thick,-] (2) -- (5);
\path[draw,thick,-] (2) -- (4);
\path[draw,thick,-] (4) -- (5);
\path[draw,thick,-] (1) -- (3);
\node at (2,-3) {vaihe #1};
}
\begin{scope}
\verkko{1}{lightgray}{white}{white}{white}{white}
\end{scope}
\begin{scope}[xshift=6cm]
\verkko{2}{lightgray}{lightgray}{white}{white}{white}
\path[draw=red,thick,->,line width=2pt] (1) -- (2);
\end{scope}
\begin{scope}[xshift=12cm]
\verkko{3}{lightgray}{lightgray}{white}{lightgray}{white}
\path[draw=red,thick,->,line width=2pt] (1) -- (2);
\path[draw=red,thick,->,line width=2pt] (2) -- (4);
\end{scope}
\begin{scope}[xshift=18cm]
\verkko{4}{lightgray}{lightgray}{white}{lightgray}{lightgray}
\path[draw=red,thick,->,line width=2pt] (1) -- (2);
\path[draw=red,thick,->,line width=2pt] (2) -- (4);
\path[draw=red,thick,->,line width=2pt] (4) -- (5);
\end{scope}
\begin{scope}[yshift=-5cm]
\verkko{5}{lightgray}{lightgray}{white}{lightgray}{lightgray}
\path[draw=red,thick,->,line width=2pt] (1) -- (2);
\path[draw=red,thick,->,line width=2pt] (2) -- (4);
\path[draw=red,thick,->,line width=2pt] (4) -- (5);
\path[draw=red,thick,->,line width=2pt] (5) -- (2);
\end{scope}
\begin{scope}[yshift=-5cm,xshift=6cm]
\verkko{6}{lightgray}{lightgray}{white}{lightgray}{lightgray}
\path[draw=red,thick,->,line width=2pt] (1) -- (2);
\path[draw=red,thick,->,line width=2pt] (2) -- (4);
\path[draw=red,thick,->,line width=2pt] (4) -- (5);
\end{scope}
\begin{scope}[yshift=-5cm,xshift=12cm]
\verkko{7}{lightgray}{lightgray}{white}{lightgray}{lightgray}
\path[draw=red,thick,->,line width=2pt] (1) -- (2);
\path[draw=red,thick,->,line width=2pt] (2) -- (4);
\end{scope}
\begin{scope}[yshift=-5cm,xshift=18cm]
\verkko{8}{lightgray}{lightgray}{white}{lightgray}{lightgray}
\path[draw=red,thick,->,line width=2pt] (1) -- (2);
\end{scope}
\begin{scope}[yshift=-10cm]
\verkko{9}{lightgray}{lightgray}{white}{lightgray}{lightgray}
\end{scope}
\begin{scope}[yshift=-10cm,xshift=6cm]
\verkko{10}{lightgray}{lightgray}{lightgray}{lightgray}{lightgray}
\path[draw=red,thick,->,line width=2pt] (1) -- (3);
\end{scope}
\begin{scope}[yshift=-10cm,xshift=12cm]
\verkko{11}{lightgray}{lightgray}{lightgray}{lightgray}{lightgray}
\end{scope}
\end{tikzpicture}
\end{center}
\caption{Esimerkki syvyyshaun toiminnasta.}
\label{fig:syvhak}
\end{figure}

Syvyyshaku on mukavaa toteuttaa rekursiivisesti seuraavaan tapaan:

\begin{code}
procedure haku(solmu)
    if vierailtu[solmu]
        return
    vierailtu[solmu] = true
    for naapuri in verkko[solmu]
        haku(naapuri)
\end{code}

Haku käynnistyy, kun kutsumme proseduuria
\texttt{haku} parametrina haun lähtö\-solmu.
Jokaisessa kutsussa proseduuri tarkistaa ensin,
onko se jo käy\-nyt parametrina annetussa solmussa,
ja päättyy heti tässä tilanteessa.
Muuten proseduuri merkitsee, että solmussa on nyt käyty,
ja etenee rekursiivisesti kaikkiin solmun naapureihin.
Haku vie aikaa $O(n+m)$, koska jokainen solmu ja kaari
käsitellään enintään kerran.

Mihin voisimme sitten käyttää syvyyshakua?
Seuraavassa on joitakin esimerkkejä syvyyshaun käyttökohteista:

\subsubsection{Polun etsiminen}

Syvyyshaun avulla voimme etsiä verkosta polun solmusta
$a$ solmuun $b$, jos tällainen polku on olemassa.
Tämä tapahtuu aloittamalla haku solmusta $a$
ja pysähtymällä, kun vastaan tulee solmu $b$.
Jos polkuja on useita, syvyyshaku löytää jonkin niistä
riippuen solmujen käsittelyjärjestyksestä.

\subsubsection{Yhtenäisyys ja komponentit}

Suuntaamaton verkko on yhtenäinen, 
jos kaikki solmut ovat yhteydessä toisiinsa.
Voimmekin tarkastaa verkon yhtenäisyyden aloittamalla
syvyyshaun jostakin solmusta ja tutkimalla,
saavuttaako haku kaikki verkon solmut.
Lisäksi voimme löytää verkon yhtenäiset komponentit
käymällä läpi solmut ja aloittamalla uuden syvyyshaun aina,
kun vastaan tulee vierailematon solmu.
Jokainen syvyyshaku muodostaa yhden komponentin.

\subsubsection{Syklin etsiminen}

\index{syklin etsiminen}

Jos suuntaamaton verkko sisältää syklin,
huomaamme tämän syvyyshaun aikana siitä,
että tulemme toista kautta johonkin solmuun,
jossa olemme käyneet jo aiemmin.
Niinpä löydämme syvyyshaun avulla jonkin verkossa olevan
syklin, jos sellainen on olemassa.

Toinen tapa tutkia syklin olemassaolo on laskea
jokaisesta verkon komponentista solmujen ja kaarten määrä.
Jos komponentissa on $x$ solmua ja siinä ei ole sykliä,
siinä tulee olla tarkalleen $x-1$ kaarta
eli komponentin tulee olla puu.
Jos kaaria on enemmän, komponentissa on varmasti sykli.

\subsection{Leveyshaku}

\begin{figure}
\center
\begin{center}
\begin{tikzpicture}[scale=0.6]
\scriptsize
\newcommand\verkko[6]{
\node[draw, circle, fill=#2] (1) at (0,0) {$1$};
\node[draw, circle, fill=#3] (2) at (2,0) {$2$};
\node[draw, circle, fill=#4] (3) at (0,-2) {$3$};
\node[draw, circle, fill=#5] (4) at (2,-2) {$4$};
\node[draw, circle, fill=#6] (5) at (4,-1) {$5$};
\path[draw,thick,-] (1) -- (2);
\path[draw,thick,-] (2) -- (5);
\path[draw,thick,-] (2) -- (4);
\path[draw,thick,-] (4) -- (5);
\path[draw,thick,-] (1) -- (3);
\node at (2,-3) {vaihe #1};
}
\begin{scope}
\verkko{1}{lightgray}{white}{white}{white}{white}
\end{scope}
\begin{scope}[xshift=6cm]
\verkko{2}{lightgray}{lightgray}{white}{white}{white}
\path[draw=red,thick,->,line width=2pt] (1) -- (2);
\end{scope}
\begin{scope}[xshift=12cm]
\verkko{3}{lightgray}{lightgray}{lightgray}{white}{white}
\path[draw=red,thick,->,line width=2pt] (1) -- (3);
\end{scope}
\begin{scope}[xshift=18cm]
\verkko{4}{lightgray}{lightgray}{lightgray}{white}{white}
\path[draw=red,thick,->,line width=2pt] (2) -- (1);
\end{scope}
\begin{scope}[yshift=-5cm,xshift=0cm]
\verkko{5}{lightgray}{lightgray}{lightgray}{lightgray}{white}
\path[draw=red,thick,->,line width=2pt] (2) -- (4);
\end{scope}
\begin{scope}[yshift=-5cm,xshift=6cm]
\verkko{6}{lightgray}{lightgray}{lightgray}{lightgray}{lightgray}
\path[draw=red,thick,->,line width=2pt] (2) -- (5);
\end{scope}
\begin{scope}[yshift=-5cm,xshift=12cm]
\verkko{7}{lightgray}{lightgray}{lightgray}{lightgray}{lightgray}
\path[draw=red,thick,->,line width=2pt] (3) -- (1);
\end{scope}
\begin{scope}[yshift=-5cm,xshift=18cm]
\verkko{8}{lightgray}{lightgray}{lightgray}{lightgray}{lightgray}
\path[draw=red,thick,->,line width=2pt] (4) -- (2);
\end{scope}
\begin{scope}[yshift=-10cm,xshift=0cm]
\verkko{9}{lightgray}{lightgray}{lightgray}{lightgray}{lightgray}
\path[draw=red,thick,->,line width=2pt] (4) -- (5);
\end{scope}
\begin{scope}[yshift=-10cm,xshift=6cm]
\verkko{10}{lightgray}{lightgray}{lightgray}{lightgray}{lightgray}
\path[draw=red,thick,->,line width=2pt] (5) -- (2);
\end{scope}
\begin{scope}[yshift=-10cm,xshift=12cm]
\verkko{11}{lightgray}{lightgray}{lightgray}{lightgray}{lightgray}
\path[draw=red,thick,->,line width=2pt] (5) -- (4);
\end{scope}
\end{tikzpicture}
\end{center}
\caption{Esimerkki leveyshaun toiminnasta.}
\label{fig:levhak}
\end{figure}

\index{leveyshaku}

\emph{Leveyshaku} (\emph{breadth-first search} eli \emph{BFS})
käy syvyyshaun tavoin läpi kaikki verkon solmut,
joihin pääsee kaaria pitkin annetusta lähtösolmusta.
Erona on kuitenkin, missä järjestyksessä solmut käydään läpi.
Leveyshaku käy solmuja läpi \emph{kerroksittain} niin,
että se käsittelee solmut siinä järjestyksessä kuin ne
ovat tulleet ensimmäistä kertaa vastaan haun aikana.

\index{lyhin polku}
\index{etäisyys}

Vaikka voisimme käyttää leveyshakua yleisenä
algoritmina verkon läpi\-käyntiin syvyyshaun tavoin,
käytämme sitä tavallisesti silloin,
kun olemme kiinnostuneita verkon \emph{lyhimmistä poluista}.
Leveyshaun avulla pystymme nimittäin määrit\-tämään
lyhimmän polun pituuden eli \emph{etäisyyden} lähtösolmusta
kuhunkin haun aikana kohtaamaamme solmuun.
Tässä oletamme, että polun pituus tarkoittaa sen
kaarten määrää eli lyhin polku on mahdollisimman vähän
kaaria sisältävä polku.

Kuvassa \ref{fig:levhak} on esimerkki leveyshaun toiminnasta,
kun aloitamme haun solmusta 1 lähtien.
Käsittelemme ensin solmun 1, josta pääsemme uusiin solmuihin 2 ja 3.
Tämä tarkoittaa, että lyhimmät polut solmuihin 2 ja 3
ovat $1 \rightarrow 2$ ja $1 \rightarrow 3$ eli etäisyys näihin solmuihin on 1.
Sitten käsittelemme solmun 2, josta pääsemme uusiin solmuihin 4 ja 5.
Tämä tarkoittaa, että lyhimmät polut solmuihin 4 ja 5
ovat $1 \rightarrow 2 \rightarrow 4$ ja $1 \rightarrow 2 \rightarrow 5$
eli etäisyys näihin solmuihin on 2.
Lopuksi käsittelemme vielä solmut 3, 4 ja 5,
joista emme kuitenkaan pääse enää uusiin solmuihin.

Tavallinen tapa toteuttaa leveyshaku on käyttää \emph{jonoa},
jossa on käsittelyä odottavia solmuja.
Jonon ansiosta pystymme käymään läpi solmut siinä
jär\-jestyksessä kuin olemme löytäneet ne leveyshaun aikana.
Oletamme, että jonossa on metodi \texttt{enqueue},
joka lisää alkion jonon loppuun, sekä metodi \texttt{dequeue},
joka hakee ja poistaa jonon ensimmäisen alkion.
Seuraava koodi suorittaa leveyshaun lähtösolmusta \texttt{alku} alkaen:

\begin{code}
jono.enqueue(alku)
vierailtu[alku] = true
etaisyys[alku] = 0
while not jono.empty()
    solmu = jono.dequeue()
    for naapuri in verkko[solmu]
        if vierailtu[naapuri]
            continue
        jono.enqueue(naapuri)
        vierailtu[naapuri] = true
        etaisyys[naapuri] = etaisyys[solmu]+1
\end{code}

Lisäämme ensin jonoon lähtösolmun ja merkitsemme,
että olemme vierailleet siinä ja että etäisyys siihen on $0$.
Tämän jälkeen alamme käsitellä solmuja siinä järjestyksessä kuin ne ovat jonossa.
Käsittelemme solmun käymällä läpi sen naapurit.
Jos emme ole aiemmin käyneet naapurissa, lisäämme sen jonoon
ja päivitämme taulukoita.
Haku vie aikaa $O(n+m)$, koska käsittelemme jokaisen
solmun ja kaaren enintään kerran.

\subsection{Esimerkki: Labyrintti}

Olemme labyrintissa ja haluamme päästä ruudusta $A$ ruutuun $B$.
Joka vuorolla voimme siirtyä yhden askeleen ylöspäin, alaspäin,
vasemmalle tai oikealle. Miten löydämme reitin?
Esimerkiksi kuvassa \ref{fig:labre1} voimme kulkea ruudusta
$A$ ruutuun $B$ reittiä, joka muodostuu 9 askeleesta.

\begin{figure}
\center
\begin{center}
\begin{tikzpicture}[scale=0.6]
\draw[fill=gray] (1,1) rectangle (2,2);
\draw[fill=gray] (1,3) rectangle (2,4);
\draw[fill=gray] (1,4) rectangle (2,5);
\draw[fill=gray] (3,1) rectangle (4,2);
\draw[fill=gray] (3,2) rectangle (4,3);
\draw[fill=gray] (3,3) rectangle (4,4);
\draw (0,0) grid (5,5);
\draw[->,thick,red,line width=2pt] (0.5,4.25) -- (0.5,2.5) -- (2.5,2.5) -- (2.5,0.5) -- (4.5,0.5) -- (4.5,1.25);
\node at (0.5,4.5) {$A$};
\node at (4.5,1.5) {$B$};
\end{tikzpicture}
\end{center}
\caption{Reitti labyrintissa ruudusta $A$ ruutuun $B$.}
\label{fig:labre1}
\end{figure}

\begin{figure}
\center
\begin{center}
\begin{tikzpicture}[scale=0.6]
\foreach \x in {0,1,2,3} \path[draw,thick,-] (0,\x) -- (0,\x+1);
\foreach \x in {0,1,2,3} \path[draw,thick,-] (2,\x) -- (2,\x+1);
\foreach \x in {0,1,2,3} \path[draw,thick,-] (4,\x) -- (4,\x+1);
\foreach \x in {0,1,2,3} \path[draw,thick,-] (\x,0) -- (\x+1,0);
\foreach \x in {0,1} \path[draw,thick,-] (\x,2) -- (\x+1,2);
\foreach \x in {2,3} \path[draw,thick,-] (\x,4) -- (\x+1,4);
\foreach \x in {0,1,2,3,4} \node[draw, circle, fill=white] at (0,\x) {};
\foreach \x in {0,2} \node[draw, circle, fill=white] at (1,\x) {};
\foreach \x in {0,1,2,3,4} \node[draw, circle, fill=white] at (2,\x) {};
\foreach \x in {0,4} \node[draw, circle, fill=white] at (3,\x) {};
\foreach \x in {0,1,2,3,4} \node[draw, circle, fill=white] at (4,\x) {};
\end{tikzpicture}
\end{center}
\caption{
Labyrintin esittäminen verkkona.}
\label{fig:labre2}
\end{figure}


Voimme esittää ongelman verkkona niin,
että jokainen lattiaruutu on yksi verkon solmuista
ja kahden solmun välillä on kaari,
jos vastaavat ruudut ovat vierekkäin labyrintissa.
Kuva \ref{fig:labre2} näyttää esimerkkilabyrinttimme verkkona.
Tätä esitystapaa käyttäen ruudusta $A$ on reitti ruutuun $B$ tarkalleen silloin,
kun vastaavat verkon solmut kuuluvat samaan yhtenäiseen komponenttiin.
Voimme käytännössä etsiä reitin syvyyshaulla tai leveyshaulla.

\index{implisiittinen verkko}

Tässä tapauksessa labyrinttia ei tarvitse erikseen muuttaa verkoksi,
vaan voimme toteuttaa haut \emph{implisiittiseen} verkkoon
eli teemme haun labyrinttiin sen omassa esitysmuodossa.
Käytännössä labyrintti on kätevää tallentaa kaksiulotteisena
taulukkona, joka kertoo, mitkä ruudut ovat seinäruutuja.

Esimerkiksi seuraava koodi toteuttaa syvyyshaun labyrintissa:

\begin{code}
procedure haku(y,x)
    if y < 0 or x < 0 or y >= n or x >= n
        return
    if seina[y][x] or vierailtu[y][x]
        return
    vierailtu[y][x] = true
    haku(y+1,x)
    haku(y-1,x)
    haku(y,x+1)
    haku(y,x-1)
\end{code}

Syvyyshaku löytää \emph{jonkin} reitin ruudusta $A$ ruutuun $B$,
mutta ei välttä\-mättä lyhintä reittiä.
Jos haluamme löytää lyhimmän reitin, meidän tulee käyttää vastaavasti leveyshakua.
